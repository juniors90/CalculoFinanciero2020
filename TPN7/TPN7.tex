\documentclass[12pt]{examdesign}
\usepackage[spanish]{babel}
\OneKey
\usepackage[utf8]{inputenc}
\usepackage[T1]{fontenc}
\usepackage{amsmath}
\usepackage{pifont}
%-----------------------------------------------------------------------------------------------
%\usepackage{gfsartemisia-euler}
\usepackage{graphicx}
\usepackage{float}
\usepackage{amscd}
\usepackage{amsfonts}
\usepackage{amssymb}
\usepackage{mathtools}
\usepackage{amsthm}
\usepackage[all]{xy}
\usepackage{enumitem}
\usepackage{multicol}
\usepackage{verbatim}
\usepackage[colorlinks=true,
breaklinks=true,
linkcolor=blue,
urlcolor=red,
bookmarksopen=true]{hyperref}
\usepackage[pdftex,dvipsnames]{xcolor}
\definecolor{aqua}{rgb}{0.0, 1.0, 1.0}
\definecolor{caribbeangreen}{rgb}{0.0, 0.8, 0.6}
\definecolor{tealgreen}{rgb}{0.0, 0.51, 0.5}
\definecolor{upforestgreen}{rgb}{0.0, 0.27, 0.13}
\definecolor{napiergreen}{rgb}{0.16, 0.5, 0.0}
\definecolor{capri}{rgb}{0.0, 0.75, 1.0}
\definecolor{calpolypomonagreen}{rgb}{0.12, 0.3, 0.17}
\definecolor{azure(colorwheel)}{rgb}{0.0, 0.5, 1.0}
\definecolor{dukeblue}{rgb}{0.0, 0.0, 0.61}
\definecolor{bole}{rgb}{0.47, 0.27, 0.23}
\definecolor{gris}{gray}{0.975}
%----------------------------------------------------------------------------------------------
% Si desea utilizar \@@line para definir su propio encabezado de examen o palabras del encabezado, 
% asegúrese de usar \makeatletter y \makeatother en los lugares apropiados, de lo contrario 
% podría obtener errores.
%-----------------------------------------------------------------------------------------------
\theoremstyle{plain}
\newtheorem{theorem}{Theorem}[section]
\newtheorem{thm}[theorem]{Teorema}
\newtheorem{cor}[theorem]{Corolario}
\newtheorem{lem}[theorem]{Lema}
\newtheorem{pro}[theorem]{Proposición}
\newtheorem{axs}[theorem]{Axiomas}
\newtheorem{axi}[theorem]{Axioma}
\theoremstyle{definition}
\newtheorem{exas}[theorem]{Ejemplos}
\newtheorem{exa}[theorem]{Ejemplo}
\newtheorem{defi}[theorem]{Definición}
\theoremstyle{remark}
\newtheorem{rmk}[theorem]{Observación}
\newtheorem{step}{Step}
\newtheorem{xca}[theorem]{Ejercicio}
\newtheorem{prob}[theorem]{Pregunta}
\newtheorem{rmks}[theorem]{Observaciones}
\newtheorem*{proofmt}{Prueba del Teorema Principal}
\usepackage[centerlast,small,sc]{caption}
\setlength{\captionmargin}{20pt}
\newcommand{\axref}[1]{Axioma~\ref{#1}}
\newcommand{\defref}[1]{\textbf{Definición}~\ref{#1}}
\newcommand{\coref}[1]{\textbf{Corolario}~\ref{#1}}
\newcommand{\thref}[1]{\textbf{Teorema}~\ref{#1}}
\newcommand{\lref}[1]{\textbf{Lema}~\ref{#1}}
\newcommand{\exaref}[1]{Ejemplo~\ref{#1}}
\newcommand{\xcaref}[1]{Ejercicio~\ref{#1}}
\newcommand{\rmkref}[1]{Observación~\ref{#1}}
\newcommand{\pref}[1]{\textbf{Proposición}~\ref{#1}}
\newcommand{\fref}[1]{Figura~\ref{#1}}
\newcommand{\tref}[1]{Tabla~\ref{#1}}
\newcommand{\cref}[1]{\textbf{Capítulo}~\ref{#1}}
\newcommand{\sref}[1]{\textbf{Sección}~\ref{#1}}
\newcommand{\aref}[1]{Apéndice~\ref{#1}}
\newcommand{\eref}[1]{Ecuación~\eqref{#1}}
\newcommand{\dref}[1]{Diagrama~\eqref{#1}}
\usepackage{makeidx}
\usepackage{tikz,tkz-tab}%
\usetikzlibrary{matrix,arrows,positioning,shadows,shadings,backgrounds,
	calc, shapes, tikzmark}
\usepackage{tcolorbox, empheq}%
\tcbuselibrary{skins,breakable,listings,theorems}

\tcbset{opteqC/.style={skin=beamer,colback=red!1!white}}
\newcommand{\celda}[2]{
	                  \begin{minipage}{#1mm}
	                      \centering
	                      \vspace{2mm}
		                      #2
		                  \vspace{2mm}
                      \end{minipage}
                     }
\makeatletter
\begin{examtop}
	\@@line{\parbox{3in}{\classdata \\[0.5cm]
			\textcolor{upforestgreen}{\textbf{\underline{T.P.N$^\circ$}}~\fbox{\textsc{7}}} \examtype}
		%                  ^^^^^^
		\hfill
		\parbox{3in}{\normalsize \namedata}}
	\bigskip
\end{examtop}

\def\namedata{\textcolor{upforestgreen}{\textbf{Estudiante}}:\hrulefill \\[\namedata@vspace]
	          \textcolor{upforestgreen}{\textbf{Curso y División}}: 6to año, I-III-IV \\[\namedata@vspace]
	          \textcolor{upforestgreen}{\textbf{Profesor}}: Ferreira, Juan David \\[\namedata@vspace]
	          \textcolor{upforestgreen}{\textbf{Fecha de Entrega}}:\hrulefill 
          }
 % manual page 11        
\begin{keytop}%
	\@@line{\hfill \Huge\texttt{\textcolor{upforestgreen}{Respuestas 
		Trabajo Práctico N$^\circ$~\fbox{\textsc{7}}}} \hfill}
	\bigskip
\end{keytop}%
\makeatother
\examname{\textcolor{upforestgreen}{\underline{\textbf{Logaritmo: Aplicación.}}}}

\SectionPrefix{\textcolor{upforestgreen}{\textbf{Sección \arabic{sectionindex}}.} \space}
\Fullpages
\ContinuousNumbering
\DefineAnswerWrapper{}{}
\NumberOfVersions{1}
\class{{\textcolor{upforestgreen}{\large\textbf{E.P.E.S. Nro 51 ``J. G. A.''}}\\[0.5cm]
		\textcolor{upforestgreen}{{\large \textbf{Gestión y Cálculo Financiero}}}}}

\begin{document}
	
	
    %-------------------------------             FILLIN       ------------------------%
    \begin{shortanswer}[title={Leemos el material de consulta y realizamos las actividades propuestas}]
    	
    	\begin{question}
    		Calcular el valor de $``x''$ en cada item, a partir de la definición logaritmo:
    		\begin{enumerate}
    			\item $\log_{4}64=x$.
    			\item $\log_{}1000=x$.
    			\item $\log_{}0,01=x$.
    			\item $\log_{2}\frac{1}{2}=x$.
    			\item $\log_{\frac{1}{2}}8=x$.
    			\item $\log_{5}125=x$.
    			\item $\log_{x}8=-3$.
    			\item $\log_{x}\frac{1}{16}=-4$.
    			\item $\log_{4}x=2$.
    			\item $\log_{4}x=0$.
    			\item $\log_{2}(x+1)=0$.
    			\item $4\cdot\log_{2}(x-1)=0$.
    			\item $\log_{2}(x+1)=3$.
    			\item $\log_{3}(8x+9)=4$.
    			\item $\log_{5}(3x-18)=3$.
    		\end{enumerate}
    	    \begin{answer}
    	    	\textbf{Respuesta:}
    	    	\begin{enumerate}
    	    		\item Usamos la definición de \textit{logaritmo} para $\log_{4}64=x\Leftrightarrow 4^{x}=64$, entonces $x=3$ porque $4^{3}=64$.
    	    		\item Usamos la definición de \textit{logaritmo} para $\log_{}1000=x\Leftrightarrow 10^{x}=1000$, entonces $x=3$ porque $10^{3}=1000$.
    	    		\item Usamos la definición de \textit{logaritmo} para $\log_{}0,01=x\Leftrightarrow 10^{x}=0,01$, aplicamoss el siguiente razonamiento
    	    		\begin{align*}
    	    			\log_{}0,01&=x\Leftrightarrow 10^{x}=0,01\mbox{ pero sabemos que $0,01=10^{-2}$}\\
    	    			\log_{}0,01&=x\Leftrightarrow 10^{x}=10^{-2} \mbox{ entonces $x=-2$}
    	    		\end{align*}
    	    		 
    	    		\item Usamos la definición de \textit{logaritmo} para $\log_{2}\frac{1}{2}=x$, y nos queda el siguiente razonamiento 
    	    		\begin{align*}
    	    		    \log_{2}\frac{1}{2}&=x\Leftrightarrow 2^{x}=\frac{1}{2}\mbox{ pero sabemos que $\frac{1}{2}=2^{-1}$}\\
    	    		    \log_{2}\frac{1}{2}&=x\Leftrightarrow 2^{x}=2^{-1} \mbox{ entonces $x=-1$}
    	    		\end{align*}.
    	    		\item $\log_{\frac{1}{2}}8=x$.
    	    		\item $\log_{5}125=x$.
    	    		\item $\log_{x}8=-3$.
    	    		\item Usamos la definición de \textit{logaritmo} para $\log_{x}\frac{1}{16}=-4$ y nos queda
    	    		\begin{align*}
    	    			\log_{x}\frac{1}{16}&=-4\Leftrightarrow x^{-4}=\frac{1}{16}\mbox{ pero sabemos que $\frac{1}{16}=16^{-1}$}\\
    	    			\log_{x}\frac{1}{16}&=-4\Leftrightarrow x^{-4}=16^{-1}\mbox{ pero sabemos que $16^{-1}={\left(2^{4}\right)}^{-1}=2^{-4}$}\\
    	    			\log_{x}\frac{1}{16}&=-4\Leftrightarrow x^{-4}=2^{-4} \mbox{ entonces $x=2$}.
    	    		\end{align*}
    	    		\item $\log_{4}x=2$.
    	    		\item $\log_{4}x=0$.
    	    		\item $\log_{2}(x+1)=0$.
    	    		\item Para resolver este logaritmo dado por $4\cdot\log_{2}(x-1)=0$, pasamos el $4$ dividiendo y nos queda
    	    		\begin{align*}
    	    		4\cdot\log_{2}(x-1)&=0\\
    	    		\log_{2}(x-1)&=0:4\\
    	    		\log_{2}(x-1)&=0\Leftrightarrow 2^{0}=x-1\mbox{ pero sabemos que $2^{0}=1$}\\
    	    		1&=x-1\mbox{ entonces $x=2$}.
    	    		\end{align*}.
    	    		\item $\log_{2}(x+1)=3$.
    	    		\item $\log_{3}(8x+9)=4$.
    	    		\item $\log_{5}(3x-18)=3$.
    	    	\end{enumerate}
    	    \end{answer}
    	\end{question}
    \end{shortanswer}
    %--------------------------------------------------------------------------------------------%
    \begin{endmatter}
    	\vspace{.2cm}
    	\centerline{\huge \textcolor{upforestgreen}{\textbf{Material de consulta acerca de Logaritmo:}}}
    	\vspace{.2cm}
    	Se llaman funciones logarítmicas a las funciones de la forma $f(a)=\log_{b}(a)$ donde $b$ se denomina \textbf{base} del logaritmo y es un número establecido distinto de 1 y mayor que 0, y ``$a$'' se denomina \textbf{argumento} del logaritmo y sus valores están comprendidos en el intervalo real $(0,\infty)$
    	\vspace{.2cm}
    	\begin{tcolorbox}[colback=red!10!white, colframe=tealgreen, title=\textbf{Logaritmo}]
    		\begin{defi}
    			El logaritmo en base $b$ de un número $a>0$ es el número $y$ que cumple la igualdad $b^{y} = a$ y se denota como $\log_{b} a$. La base ``$b$'' debe ser un número real positivo distinto de $1$. Es decir,
    			\begin{equation}
    			     y=\log_{b}a \Longleftrightarrow b^{y} = a,\mbox{ con } b\not=1 \mbox{ y } a>0.
    			\end{equation}
    		     El número ``$a$'' recibe el nombre de \textbf{argumento} del logaritmo.
    		\end{defi}
    	\end{tcolorbox}
    	\vspace{.2cm}
    	Las bases $b\in \mathbb{N}$ ($\mathbb{N}$ es el conjunto de los números naturales) que más se utilizan en los logaritmos son $10$ y $2$. Por esta razón, solemos referirnos a ellos directamente como logaritmo decimal y logaritmo binario, respectivamente.
    	\vspace{.2cm}
    	\begin{tcolorbox}[opteqC]
    		\begin{exa}
    			Decimos que $y$ es el logaritmo decimal del número $a$ es si cumple que es el logaritmo en base $10$ de $a$. Es decir,
    			\begin{equation}
    			    y=\log_{10}a \Longleftrightarrow 10^{y} = a.
    			\end{equation}
    			Cuando hablamos de logaritmo decimal, simplemente omitimos escribir la base, es decir:
    			\begin{equation}
    			    y=\log a \Longleftrightarrow 10^{y} = a.
    			\end{equation}
    			Luego podemos calcular algunos logaritmos decimales usando la definicion,
    			\begin{align*}
    			    2=\log 100 &\Longleftrightarrow 10^{2} = 100&y&&3=\log 1000 &\Longleftrightarrow 10^{3} = 1000.
    			\end{align*} 
    		\end{exa}
        \end{tcolorbox}
        \vspace{.2cm}
        \begin{tcolorbox}[opteqC]
        	\begin{exa}
        		Decimos que $y$ es el logaritmo binario del número $a$ es si cumple que es el logaritmo en base $2$ de $a$. Es decir,
        		\begin{equation}
        		    y=\log_{2}a \Longleftrightarrow 2^{y} = a.
        		\end{equation}
        		Luego podemos calcular algunos logaritmos binarios usando la definicion,
        		\begin{align*}
        		    3=\log_{2}8 &\Longleftrightarrow 2^{3} = 8&y&&4=\log_{2}16 &\Longleftrightarrow 2^{4} = 16
        		\end{align*}
        	\end{exa}
        \end{tcolorbox}
 
    \end{endmatter}
\end{document}