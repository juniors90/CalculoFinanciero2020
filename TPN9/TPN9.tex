\documentclass[12pt]{examdesign}
\usepackage[spanish]{babel}
\OneKey
\usepackage[utf8]{inputenc}
\usepackage[T1]{fontenc}
\usepackage{amsmath}
\usepackage{pifont}
\usepackage{graphicx}
\usepackage{float}
\usepackage{amscd}
\usepackage{amsfonts}
\usepackage{amssymb}
\usepackage{mathtools}
\usepackage{amsthm}
\usepackage{enumitem}
\usepackage{multicol}
\usepackage{verbatim}
\usepackage[colorlinks=true,
breaklinks=true,
linkcolor=blue,
urlcolor=red,
bookmarksopen=true]{hyperref}
\usepackage[pdftex,dvipsnames]{xcolor}
\definecolor{aqua}{rgb}{0.0, 1.0, 1.0}
\definecolor{caribbeangreen}{rgb}{0.0, 0.8, 0.6}
\definecolor{tealgreen}{rgb}{0.0, 0.51, 0.5}
\definecolor{upforestgreen}{rgb}{0.0, 0.27, 0.13}
\definecolor{napiergreen}{rgb}{0.16, 0.5, 0.0}
\definecolor{capri}{rgb}{0.0, 0.75, 1.0}
\definecolor{calpolypomonagreen}{rgb}{0.12, 0.3, 0.17}
\definecolor{azure(colorwheel)}{rgb}{0.0, 0.5, 1.0}
\definecolor{dukeblue}{rgb}{0.0, 0.0, 0.61}
\definecolor{bole}{rgb}{0.47, 0.27, 0.23}
\definecolor{gris}{gray}{0.975}
%----------------------------------------------------------------------------------------------
% Si desea utilizar \@@line para definir su propio encabezado de examen o palabras del encabezado, 
% asegúrese de usar \makeatletter y \makeatother en los lugares apropiados, de lo contrario 
% podría obtener errores.
%-----------------------------------------------------------------------------------------------
\theoremstyle{plain}
\newtheorem{theorem}{Theorem}[section]
\newtheorem{thm}[theorem]{Teorema}
\newtheorem{cor}[theorem]{Corolario}
\newtheorem{lem}[theorem]{Lema}
\newtheorem{pro}[theorem]{Proposición}
\newtheorem{axs}[theorem]{Axiomas}
\newtheorem{axi}[theorem]{Axioma}
\theoremstyle{definition}
\newtheorem{exas}[theorem]{Ejemplos}
\newtheorem{exa}[theorem]{Ejemplo}
\newtheorem{defi}[theorem]{Definición}
\theoremstyle{remark}
\newtheorem{rmk}[theorem]{Observación}
\newtheorem{step}{Step}
\newtheorem{xca}[theorem]{Ejercicio}
\newtheorem{prob}[theorem]{Pregunta}
\newtheorem{rmks}[theorem]{Observaciones}
\newtheorem*{proofmt}{Prueba del Teorema Principal}
\usepackage[centerlast,small,sc]{caption}
\setlength{\captionmargin}{20pt}
\newcommand{\axref}[1]{Axioma~\ref{#1}}
\newcommand{\defref}[1]{\textbf{Definición}~\ref{#1}}
\newcommand{\coref}[1]{\textbf{Corolario}~\ref{#1}}
\newcommand{\thref}[1]{\textbf{Teorema}~\ref{#1}}
\newcommand{\lref}[1]{\textbf{Lema}~\ref{#1}}
\newcommand{\exaref}[1]{Ejemplo~\ref{#1}}
\newcommand{\xcaref}[1]{Ejercicio~\ref{#1}}
\newcommand{\rmkref}[1]{Observación~\ref{#1}}
\newcommand{\pref}[1]{\textbf{Proposición}~\ref{#1}}
\newcommand{\fref}[1]{Figura~\ref{#1}}
\newcommand{\tref}[1]{Tabla~\ref{#1}}
\newcommand{\cref}[1]{\textbf{Capítulo}~\ref{#1}}
\newcommand{\sref}[1]{\textbf{Sección}~\ref{#1}}
\newcommand{\aref}[1]{Apéndice~\ref{#1}}
\newcommand{\eref}[1]{Ecuación~\eqref{#1}}
\newcommand{\dref}[1]{Diagrama~\eqref{#1}}
\usepackage{makeidx}
\usepackage{tikz,tkz-tab}%
\usetikzlibrary{matrix,arrows,positioning,shadows,shadings,backgrounds,
	calc, shapes, tikzmark}
\usepackage{tcolorbox, empheq}%
\tcbuselibrary{skins,breakable,listings,theorems}

\tcbset{opteqC/.style={skin=beamer,colback=red!1!white}}

\makeatletter
\begin{examtop}
	\@@line{\parbox{3in}{\classdata \\[0.5cm]
			\textcolor{upforestgreen}{\textbf{\underline{T.P.N$^\circ$}}~\fbox{\textsc{9}}} \examtype}
		%                  ^^^^^^
		\hfill
		\parbox{3in}{\normalsize \namedata}}
	\bigskip
\end{examtop}

\def\namedata{\textcolor{upforestgreen}{\textbf{Estudiante}}:\hrulefill \\[\namedata@vspace]
	          \textcolor{upforestgreen}{\textbf{Curso y División}}: 6to año, I-III-IV \\[\namedata@vspace]
	          \textcolor{upforestgreen}{\textbf{Profesor}}: Ferreira, Juan David \\[\namedata@vspace]
	          \textcolor{upforestgreen}{\textbf{Fecha de Entrega}}:\hrulefill 
          }
 % manual page 11        
\begin{keytop}%
	\@@line{\hfill \Huge\texttt{\textcolor{upforestgreen}{Respuestas 
		Trabajo Práctico N$^\circ$~\fbox{\textsc{9}}}} \hfill}
	\bigskip
\end{keytop}%
\makeatother
\examname{\textcolor{upforestgreen}{\underline{\textbf{Interés y Capital.(Parte 2)}}}}

\SectionPrefix{\textcolor{upforestgreen}{\textbf{Sección \arabic{sectionindex}}.} \space}
\Fullpages
\ContinuousNumbering
\DefineAnswerWrapper{}{}
\NumberOfVersions{1}
\class{{\textcolor{upforestgreen}{\large\textbf{E.P.E.S. Nro 51 ``J. G. A.''}}\\[0.5cm]
		\textcolor{upforestgreen}{{\large \textbf{Gestión y Cálculo Financiero}}}}}

\begin{document}
    %------------------------------------                  shortanswer                   -------------------------------------%
    \begin{shortanswer}[title={Leemos el material de consulta y realizamos las actividades propuestas.}, rearrange=no]
    	\begin{question}
    		Calcular el interés que generan $\$400.000$ durante $4$ meses a un tipo de interés anual del $30\%$.
    	    \begin{answer}
    	    	%% ===================================================================================================== %%
    	    	%% ========================= RESPUESTA AL EJERCICIO 1 - ÍTEM A) - INTERÉS SIMPLE    ==================== %%
    	    	%% ===================================================================================================== %%
    	    	Para calcular el interés que generan $\$400.000$ durante $4$ meses a un tipo de interés anual del $30\%$, primero identificalos los datos.
    	    	
    	    	\textbf{Datos:}
    	    	
    	    	\begin{itemize}
    	    		\item $C_{0}=$
    	    		\item $r=$
    	    		\item $n=$
    	    		\item $I=$
    	    	\end{itemize}
        	
    	    	\textbf{Respuesta:} El interés que generan $\$400.000$ durante $4$ meses a un tipo de interés anual del $30\%$.
    	    \end{answer}
    	\end{question}
        
        \begin{question}
        	Calcular el capital final que tendríamos si invertimos $\$100000$ durante $6$ meses al $42\%$ de interes anual.	
        	\begin{answer}
        		%% ===================================================================================================== %%
        		%% =========================   RESPUESTA AL EJERCICIO 2 - ÍTEM A) - INTERÉS SIMPLE  ==================== %%
        		%% ===================================================================================================== %%
        		\textbf{Respuesta:} Para calcular el capital final que tendríamos si invertimos $\$100000$ durante $6$ meses al $42\%$ de interes anual..
        			    
        	    \textbf{Datos:}
        	    
        	    \begin{itemize}
        	    	\item $C_{0}=$
        	    	\item $r=$
        	    	\item $n=$
        	    	\item $I=$
        	    \end{itemize}
        	\end{answer}
        \end{question}
        
        \begin{question}
        	Recibimos $\$50000$ dentro de $6$ meses y $\$60000$ dentro de $9$ meses, y ambas cantidades las invertimos a un tipo del $32\%$ anual. Calcular que importe tendríamos dentro de $1$ año.
        	\begin{answer}
        		%% ===================================================================================================== %%
        		%% =========================   RESPUESTA AL EJERCICIO 2 - ÍTEM A) - INTERÉS SIMPLE  ==================== %%
        		%% ===================================================================================================== %%
        		\textbf{Respuesta:} Si recibimos $\$50000$ dentro de $6$ meses y $\$60000$ dentro de $9$ meses, y ambas cantidades las invertimos a un tipo del $32\%$ anual y queremos calcular que importe tendríamos dentro de $1$ año, debemos identificar los datos.
        		
        		\textbf{Datos:}
        		\begin{itemize}
        			\item $C_{0}=$
        			\item $r=$
        			\item $n=$
        			\item $I=$
        		\end{itemize}
        	\end{answer}
        \end{question}
        
        \begin{question}
        	Calcular los intereses devengados por una capitalización de $\$4000$, durante 3 meses, a un tipo de interés del $23\%$ anual.
        	\begin{answer}
        		\textbf{Respuesta:}
        		
        		\textbf{Datos:}
        		\begin{itemize}
        			\item $C_{0}=$
        			\item $r=$
        			\item $n=$
        			\item $I=$
        		\end{itemize}
        	\end{answer}
        \end{question}
    
        \begin{question}
        	Calcular los intereses devengados por una capitalazación de $20000$, durante 1 año, a un tipo de interés del $10\%$ mensual.
        	\begin{answer}
        		\textbf{Respuesta:}
        		
        		\textbf{Datos:}
        		\begin{itemize}
        			\item $C_{0}=$
        			\item $r=$
        			\item $n=$
        			\item $I=$
        		\end{itemize}
        	\end{answer}
        \end{question}
    \end{shortanswer}
    %---------------------------------------------------------------------------------------------------------------------------------------------------------------%
    \begin{endmatter}
    	
    	\centerline{\LARGE \textcolor{upforestgreen}{\textbf{Material de consulta: Interés Y Capital Financiero}}}
    	\vspace{.05cm}
    	\noindent \textbf{Interés Y Capital Financiero}
    	\vspace{.1cm}
    	
    	\begin{tcolorbox}[colback=red!10!white, colframe=tealgreen, title=\textbf{Los principales conceptos a trabajar serán son los siguientes:}]
    		\begin{itemize}
    			\item \textbf{Capital Inicial}: $C_{0}$, cantidad de dinero invertida susceptible de sufrir una variación cuantitativa.
    			\item \textbf{Tasa de Interés}: es el interés producido por $\$100$ en una unidad de tiempo determinada, representado por un porcentaje, $i$, que produce la variación en la unidad de tiempo. Se suele usar en las fórmulas al rédito, $r$, que es la tasa de interés expresada de manera decimal.
    			\item \textbf{Período}: $n$, cantidad de tiempo por el cual se coloca el capital inicial.
    			\item \textbf{Monto}: $M$, ó $C_{n}$, cantidad final monetaria, ó capital final luego de un tiempo $n$, que se obtiene al finalizar el periodo en que se generan los intereses.
    			\begin{equation}
    			    M = C_{0} + I = C_{n}
    			\end{equation}
    		\end{itemize}
    	\end{tcolorbox}
    	\vspace{.2cm}
    	
    	Las operaciones financieras que trabajaremos se clasifican según la Ley Financiera que opera generando intereses a:
    	
    	\begin{itemize}
    		\item Interés Simple: todos los intereses son calculados sobre el capital inicial.
    		\item Interés Compuesto: los intereses generados se acumulan al capital inicial generando así otros intereses.
    	\end{itemize}
    	
    	\vspace{.2cm}
    	\begin{tcolorbox}[opteqC]
    		\begin{exa}
    		 Luis le prestó $\$800$ a Julia a devolver al cabo de un año con una tasa de interés del $5\%$ trimestral. ¿Cuál será el interés generado en ese tiempo? ¿Cuánto dinero le devolverá Julia a Luis?
    		\end{exa}
        \end{tcolorbox}
        Si:
        \begin{align}
        	I&=C_{0}\cdot r\cdot n\\
        	I&=800\cdot \frac{5}{100}\cdot 4=160\\
        	M&=C_{0}+I\\
        	M&=800+160=960
        \end{align}
        O sea que los intereses generado en $4$ trimestres (1 año), con una tasa de interés del $5\%$ por trimestre, aplicado a $\$800$ serán $\$160$. Y Julia deberá devolver al final de ese tiempo $\$960$
        .
        De acuerdo a la situación anterior encuentra una expresión que relacione el monto en función de los datos del problema:
        \begin{center}
        	\def\blank#1{$\underline{\mbox{\hphantom{#1}}}$}
        	$M$=\blank{------------------}
        \end{center}
        \vspace{.2cm}
        \begin{tcolorbox}[opteqC]
        	\begin{exa}
        		Luis le prestó $\$800$ a Julia a devolver al cabo de un año con una tasa de interés del $5\%$ trimestral. ¿Cuál será el interés generado en ese tiempo? ¿Cuánto dinero le devolverá Julia a Luis?
        	\end{exa}
        \end{tcolorbox}
        \begin{align*}
        	&\mbox{En el primer trimestre: }               &&&        I&=800\cdot\frac{5}{100}=40          \\
        	&                                              &&&      C_1&=C_{0}+I=840                       \\
        	&\mbox{En el segundo trimestre: }              &&&        I&=840\cdot \frac{5}{100}=42         \\
        	&                                              &&&      C_2&=C_{1}+I=882                       \\
        	&\mbox{En el tercer trimestre: }               &&&        I&=882\cdot \frac{5}{100}=44,10      \\
        	&                                              &&&      C_3&=C_{2}+I=926,10                    \\
        	&\mbox{Por último, en el cuarto trimestre: }   &&&        I&=926,10 \cdot \frac{5}{100}=46,305 \\
        	&                                              &&&    C_{4}&=C_{3}+I=972,405                   \\
        	&                                              &&&        M&=C_4=972.405=800+I                 \\ 
        	&                                              &&&        I&=172,405
        \end{align*}
        O sea que los intereses generado en 4 trimestres (1 año), con una tasa de interés del $5\%$ por trimestre, aplicado a $\$800$ serán $\$172.405$. Y Julia deberá devolver al final de ese tiempo $\$972,405$.
        
        Por lo tanto la expresión que permite calcular el capital final (monto) en un interés compuesto de acuerdo a los datos del problema es:
        
        \begin{center}
        	\def\blank#1{$\underline{\mbox{\hphantom{#1}}}$}
        	$M$=\blank{------------------}
        \end{center}
        \vspace{.1cm}
        \begin{tcolorbox}[colback=red!10!white, colframe=tealgreen, title=\textbf{Resumen de las Formulas:}]
        	\begin{itemize}
        		\item Interés Simple
        		\begin{align*}
        		    I &= C_{0}\cdot r\cdot n
        		\end{align*}
        		\item \textbf{Monto}: $M$ = cantidad final monetaria ó $C_{n}$ = capital final luego de un tiempo $n$, que se obtiene al finalizar el periodo en que se generan los intereses.
        		\begin{equation}
        		C_{f}= M = C_{0} + I = C_{n}
        		\end{equation}
        		\item Interés compuesto:
        			\begin{equation}
        		        C_{f}= M = C_{0}\left(1+r\right)^{n}
        		    \end{equation}
        	\end{itemize}
        \end{tcolorbox}
    \end{endmatter}
\end{document}